\subsection{The \texttt{GenObf} Function}

Now, we are ready to present the details of the  \texttt{GenObf()} function for finding 
a $(k,\epsilon)$-obf instance for an input uncertain graph $\mathcal{G}$ in Algorithm \ref{alg:genObf}. The function receives the parameters that are originally passed to \SysName skeleton (Algorithm~\ref{alg:Skeleton}) including noise parameter $\sigma$.  



\textbf{Uniquness \& Relevance Computation.}~~The function begins the computation of the uniqueness score and reliability relevance. (Lines 1 \& 2).
These two invariants correspond to our goals of preserving privacy \& utility.   
Based on these weighting factors, the \texttt{GenObf} then heuristically performs edge selection \& perturbation, i.e, use the noise budget in the most effective way. 


\textbf{Exclusion.}~~Since it is allowed not to obfuscate $\epsilon|V|$ of the nodes per the problem definition, the algorithm leverages the two invariants highlighted above and selects a set $H$ of $\frac{\epsilon}{2}|V|$ nodes with the largest combined uniqueness and reliability relevance scores, and excludes them from subsequent obfuscation efforts. 

\begin{algorithm}[!htb]
% {\scriptsize
	\begin{algorithmic}[1]
    	\item[] {\textbf{Input:}~Uncertain graph $\mathcal{G}=(V,E,\mathit{p})$, $\mathcal{K},k,\epsilon,c,q$, \\and standard deviation $\sigma$ }
        \item[] {\textbf{Output:}~A pair $\langle \tilde{\epsilon}, \tilde{\mathcal{G}} \rangle$} where $\tilde{\mathcal{G}}$ is a $(k,\epsilon)-$obfuscation, or $\tilde{\epsilon}=1$ if fail to find a $\keobf$. 
        \STATE  {\textbf{compute} the uniqueness $U^{v}$ for $v \in V$}
        \STATE  {\textbf{compute} the reliability relevance $\mathcal{V}RR^{v}$ for $v \in V$}
        \STATE  {$Q^{v} \leftarrow U^{v} \cdot \mathcal{V}RR^{v}$ for $v \in V$}
        \STATE  {$H \leftarrow$  the set of $\lceil \frac{\epsilon}{2} |V| \rceil$ with largest $Q^{v}$}
     	\STATE  {Normalized $\mathcal{V}RR^{v}$ for $v \in V \setminus H$}
        \STATE  {$Q^{v} \leftarrow U^{v} \cdot 1-\mathcal{V}RR^{v}$ for $v \in V \setminus H$}
        \STATE {$\tilde{\epsilon} \leftarrow 1$}
   		\FOR{$t$ times} 
%         	\COMMENT{\textcolor{blue} {\scriptsize $\nabla$~~Find an initial successful obfuscation}}
         	\REPEAT  
                \STATE {$E_{C} \leftarrow E$} 
            	\STATE{randomly pick a vertex $u \in V \setminus H$ according to $Q$}
            	\STATE{randomly pick a vertex $v \in V \setminus H$ according to $Q$}
            	% \STATE{draw $w$ uniformly at random from $[0,1]$} 
                % \IF {$(u,v) \in E$} 
                %     \STATE {$E_{C} \leftarrow E_{c} \setminus \lbrace(u,v)\rbrace$ with the probability $p(e)$}
                % \ELSE 
                %     \STATE{$E_{c} \leftarrow E_{c} \cup \lbrace(u,v)\rbrace$}
                % \ENDIF 
                \STATE{{\bf if} $(u,v) \in E$}
                \STATE{{\bf then} $E_{C} \leftarrow E_{c} \setminus \lbrace(u,v)\rbrace$ with the probability $p(e)$}
                \STATE{{\bf else}~$E_{c} \leftarrow E_{c} \cup \lbrace(u,v)\rbrace$} 
            \UNTIL{$E_{C}=c|E|$}
            \FORALL {$e \in E_{C}$} 
            	\STATE {\textbf{compute} $\sigma(e)$}
                \STATE {draw $w$ uniformly at random from $[0,1]$}
                \STATE {{\bf if} $w <q$~{\bf then} $r_{e} \leftarrow U(0,1)$ }
                \STATE {{\bf else} $r_{e} \leftarrow R_{\sigma(e)}$}
				% \IF {$w < q$} \STATE{$r_{e} \leftarrow U(0,1)$}
    %             \ELSE 
    %             \STATE{$r_{e} \leftarrow R_{\sigma(e)}$}
    %             \ENDIF
                \STATE \textbf{$\hat{p}(e) \leftarrow p(e)+ (1-2p(e))\cdot r_{e}$}
            \ENDFOR
            \STATE {$\hat{\epsilon} \leftarrow \text{anonymityCheck}(\tilde{\mathcal{G}})$} 
            % \IF{$\hat{\epsilon}<\epsilon$ and $\hat{\epsilon}< \tilde{\epsilon}$} 
            % \STATE{$\tilde{\epsilon} \leftarrow \hat{\epsilon}$; $\tilde{\mathcal{G}} \leftarrow \hat{\mathcal{G}}$}
            % \ENDIF
            \STATE {{\bf if}~$\hat{\epsilon}<\epsilon$ and $\hat{\epsilon}$~{\bf then} $\tilde{\epsilon} \leftarrow \hat{\epsilon}$; $\tilde{\mathcal{G}} \leftarrow \hat{\mathcal{G}}$}
        \ENDFOR 
        \STATE {return $\langle \tilde{\epsilon}, \tilde{\mathcal{G}} \rangle$}
      	\caption{GenObf}
        \label{alg:genObf}
    \end{algorithmic}
    }
\end{algorithm}

\textbf{Unifying Uniqueness and Relevance Score.}~~
Nodes not in $H$ are candidates for anonymization. 
To anonymize high-uniqueness vertices, higher noise needs to be injected. Thus, edges associated with those vertices need to be sampled with a higher probability. Meanwhile, to better preserve the graph structure, edges associated with high reliability-relevance nodes need to be sampled with a smaller probability.
In order to implement such sampling strategy, our algorithm assigns a probability $Q^{v}$ to every $v \in V \setminus H$ ($v$ in $V$ but not in $H$), 
which is proportional to $v$'s uniqueness $U^{v}$ and inverse proportional to $v$'s reliability relevance $\mathcal{V}RR^{v}$. 

 

\textbf{Edge Selection.}~~After that, the algorithm starts its $t$ trials for finding $\keobf$. Each trial first selects a set of candidate edges $E_{c}$, which will be subject to probability perturbation.
Initially $E_{c}$ is set to $E$. Then, the algorithm randomly selects two distinct vertices $u$ and $v$, according to their assigned probabilities. 
The edge $(u,v)$ is then excluded from $E_{c}$ with the probability $p(e)$ if it is an edge in the original graph (Lines 14), 
otherwise it is added to $E_{c}$ (Line 15).  
The process is repeated until $E_{c}$ reaches the required size, which is controlled by the input parameter $c$.
In typical uncertain graphs, the number of absent edges is usually significantly larger than the number of present uncertain edges. 
Thus, the loop usually ends very quickly for small values of $c$. And, the resulting set $E_{c}$ includes most of edges in $E$. 

\textbf{Edge Perturbation.}~~Next, we redistribute the noise budgets among all selected edges in proportion to their unify weighting factors. pecially, we define for each $e=(u,v) \in E_{c}$ , its uncertainty level, 
\begin{equation*}
    Q^{e}:= \frac{Q^{u}+Q^{v}}{2}
    % \vspace{-1em}
\end{equation*}
and then set  
\begin{equation*}
    \sigma(e)=\sigma |E_{c}|  \cdot \frac{Q^{e}}{\sum_{e \in E_{c}} Q^{e}}
    % \vspace{-1em}
\end{equation*}
so that the average of $\sigma(e)$ over all $e \in E_{C}$ equals $\sigma$.

\textbf{Edge Probability Perturbation.}~~If we carefully perform edge prob alteration with the edge uncertainty levels, $\sigma(e)$, we effectively obfuscate node. In the following section, we will instantiate our ideas.

\textbf{Success or Failure.} Finally, If the algorithm successfully finds $(k,\epsilon)$-obfuscated graph in one of its $t$ trials, it returns the obfuscated graph with minimal $\epsilon$. Otherwise, it indicates the failure by returning $\tilde{\epsilon}=1$. 
