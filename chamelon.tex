\section{Privacy Via XXX}
\label{sec:tech}
Instead of detaching edge uncertainty from the anonymization phase, we shift the state-of-art method by integrating uncertainty semantics into its core steps, namely XXX.
It enables a unifying and grained control over the noise injected to uncertain graphs, then provides enough privacy guarantee with good utility.

\subsection{The State-of-Art Framework}
\textbf{Problem Transformation}~~Anonymization is done via altering the probabilities of sampled edges. For each sampled edge $e$, it is assigned a probability deviation $r_{e}$, where $r_{e} \leftarrow R(\sigma)$. As shown in the work~\cite{Boldi_Injecting_2012}, the distribution $R(\sigma)$ is a truncated normal distribution with mean 0 and variance $\sigma^2$. But, it could in principle be any distribution. 

As the standard deviation $\sigma$ decreases, a greater mass of $R_{\sigma}$ will concentrate near $r_{e}=0$, then the amount of injected noise and consequent structural deviation will be smaller. It enables transforming the graph anonymization problem into the minimization of structural noise need. The later one can be achieved via a binary search on the value of standard deviation $\sigma$.
\begin{algorithm}
{\scriptsize
	\begin{algorithmic}[1]
    	\item[] {\textbf{Input:}~Graph $\mathcal{G}$, obfuscation level $k$, tolerance parameter $\epsilon$}
        \item[] {\textbf{Output:}~The result $\mathcal{G}_{obf}$}
     	\STATE {$\sigma_{l} \leftarrow 0$; $\sigma_{u} \leftarrow 1$} \\
        \REPEAT
        \STATE{$\langle \hat{\epsilon}, \hat{\mathcal{G}} \rangle$ $\leftarrow$ \textbf{genObf}(-,$\sigma_{u}$)} \\
        \STATE{{\bf if} $\hat{\epsilon}=1$ (fail) {\bf then} $\sigma_{l} \leftarrow \sigma_{u}$; $\sigma_{u} \leftarrow 2\sigma_{u}$}
        \UNTIL{$\hat{\epsilon} \neq 1 $} \\
        \REPEAT
        	\STATE {$\sigma_{mid} \leftarrow (\sigma_{u}+\sigma_{l})/2$}
            \STATE{$\langle \hat{\epsilon}, \hat{\mathcal{G}} \rangle$ $\leftarrow$ \textbf{genObf}(-,$\sigma_{mid}$)}
            \STATE {{\bf if} $\hat{\epsilon} =1$~{\bf then}~$\sigma_{l} \leftarrow \sigma_{mid}$}\\
            \STATE {{\bf else} $\sigma_{u} \leftarrow \sigma_{mid}$;~~{$\mathcal{G}}_{obf} \leftarrow \hat{\mathcal{G}}$}
        \UNTIL{$\sigma_{u}-\sigma_{l}$ is enough small}
        % \COMMENT{\textcolor{blue}{\scriptsize Binary search for better obfuscation}}
        \STATE {return $\mathcal{G}_{obf}$}
    	\caption{The obfuscation algorithm}
	 \label{alg:obf}
    \end{algorithmic}
    }
\end{algorithm}
\vspace{-7pt}


\textbf{Search Flow}~~The binary search flow is determined by the \textmd{GenObf} function. For a given value of standard deviation $\sigma$, \textmd{GenObf} either returns the best found {\keobf} instance or indicates failure. 
The binary-search process starts with an initial guess of an upper bound $\sigma_{u}$ which is iteratively doubled until a ${\keobf}$ graph is found. Then, it is performed using $\sigma_{l}$ = 0 as the lower bound, and the found upper bound $\sigma_{u}$. The binary search terminates when the search interval is sufficiently short. It outputs the best {\keobf} found (i.e., the
 last one that was successfully generated).



